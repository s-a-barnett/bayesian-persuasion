\documentclass{article}
\usepackage[utf8]{inputenc}
\DeclareUnicodeCharacter{FB01}{fi}
\usepackage{amsmath}
\usepackage{amssymb}
\usepackage{bm}

\title{A Rational Analysis of Persuasion}
\author{S. A. Barnett}
\date{}



\usepackage[a4paper, total={6in, 8in}]{geometry}

\usepackage[
backend=biber,
style=numeric,
sorting=ynt
]{biblatex}
\addbibresource{proposal.bib}

\begin{document}

\maketitle

\section{Background}
%\textit{Identify the topic your project will explore, and briefly provide some of the context for your project, describing what previous research has found in this area.}

This project will investigate inference within a social context: namely, given that the evidence that an individual receives originates from human communication, how does our response to this evidence differ from our normative expectations about how such inferences ought to be drawn? In particular, I will focus on the context in which a person hears two different sides of a debate, and must update their beliefs based on the evidence provided each side. 

My aim is to provide a computational model of inference over a debate based upon Bayesian principles. This is inspired by the Rational Speech Act model of communication \cite{goodman_pragmatic_2016}, in which language understanding is viewed as a form of Bayesian inference whose sampling procedure is influenced by having a theory of mind for the speaker. However, rather than the cooperative principle underpinning the inference model \cite{grice_logic_1975}, the debate context requires that a pragmatic listener instead models each speaker as aiming to be maximally \textit{persuasive}: I call the principle that governs such speech acts as the \textit{rhetorical principle.} Hence, my project will provide a model whose inference procedure embodies this principle.

Previous research has found that listeners in a debate context display two patterns of inference: the \textit{weak evidence effect} \cite{mckenzie_when_2002, fernbach_when_2011, harris_james_2013}, and the \textit{strong evidence effect} \cite{perfors_stronger_2018}. The \textit{weak evidence effect} occurs when receiving weak evidence makes people less likely to believe a conclusion relative to the marginal belief. This effect may arise from receivers expecting debaters to produce the strongest evidence for their case. The \textit{strong evidence effect}, on the other hand, occurs when strong evidence does not always lead to stronger inferences. This effect may arise from the possibility that stronger evidence makes the listener believe that the debater was overly biased, causing the listener to discount the evidence.

\section{Question}
%\textit{State the specific question you are going to examine in your final project.}
Can the weak and strong evidence effects be accounted for by a rhetorical principle?

\section{Method}
%\textit{Briefly describe the method you are going to use to try to answer this question. Give some of the details behind your experimental procedure, your approach to modeling or analyzing the data, your plans for analyzing the model, or the position you will take in your review.}

The experiments are run with the \textit{sticks game}. In this game, a sample of $N$ sticks is drawn, with the stick length being modelled as i.i.d. draws from a standard uniform distribution: \[ \{ U_n\}_{n=1}^N \sim \mathcal{U}[0, 1] .\] Two agents observe all of the sticks in the sample, and at each time step the agents take it in turn to show a stick to the judge. The judge must decide whether the sample is `long': that is, whether \[ \bar{U} := \frac{1}{N} \sum_{n=1}^N U_n \ge 0.5 .\]

\begin{description}
\item[Experiment 1] The human plays the judge, and observes two time steps of the game for different values of $N$. At each step, the human is asked to report her confidence that the sample is `long'. These confidence ratings are contrasted with the conditional probability evaluations (where the data is assumed to be drawn i.i.d.) to test for the weak and strong evidence effects.

\item[Experiment 2] Two different models of the judge are evaluated on the same task: one in which the judge \textit{assumes} that each agent is sampling the longest and shortest sticks (respectively), and one in which the agent has uncertainty over how the agents are biased (including whether they are biased at all). The plot of the posterior belief in the sample length is used to test for the weak and strong evidence effects.

 \item[Experiment 3] The agents are biased, and modelled in order to be maximally persuasive to a judge with uncertainty over the bias of the speaker. Concretely, at each step they select the stick that minimizes the total variation distance between having full confidence in the view consistent with the agent's bias, and the posterior determined by the judge in the above model. The intention of this experiment is to model the strong evidence effect in the speakers by investigating their action choices.
\end{description}

\printbibliography
\end{document}